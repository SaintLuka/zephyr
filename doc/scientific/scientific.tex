\documentclass[12pt,draft]{report}
\usepackage[left=20mm, top=20mm, right=20mm, bottom=20mm, nohead]{geometry}
\usepackage[T2A]{fontenc}
\usepackage[utf8x]{inputenc}
\usepackage[english, russian]{babel}
\usepackage{graphicx}
\usepackage{cite}
\usepackage{amssymb,amsmath,nccmath,latexsym,enumerate}
\usepackage{physics}
\usepackage{array,longtable,lscape}
\usepackage{graphicx}
\usepackage[usenames]{color}
\usepackage{titlesec}
\usepackage{comment}
\usepackage{arydshln}
\usepackage{dashrule}
\usepackage{setspace} 
\usepackage{physics}
\usepackage{indentfirst}


\numberwithin{equation}{chapter}

\newcommand{\eps}{\varepsilon}
\newcommand{\ups}{\upsilon}
\newcommand{\half}{\frac{1}{2}}
\newcommand{\thalf}{\tfrac{1}{2}}
\newcommand{\const}{\mathrm{const}}
\newcommand{\class}[1]{\texttt{\textcolor{cyan}{#1}}}
\newcommand{\namespace}[1]{\texttt{\textcolor{blue}{#1}}}
\newcommand{\file}[1]{\texttt{#1}}
\DeclareMathOperator{\sgn}{sgn}

\bibliographystyle{plain}

\onehalfspacing


\begin{document}

\renewcommand{\contentsname}{Научная документация}
\tableofcontents

\chapter{Уравнения состояния}

\section{Краткие сведения из термодинамики}

Каноническим уравнением состояния называется зависимость одного из четырех термодинамических потенциалов от пары своих естественных переменных:
\begin{center}
\renewcommand{\arraystretch}{1.3}
\begin{tabular}{lccl}
    $U$ &=& $U(S, \, V) \quad $ & --- внутренняя энергия, \\
    $H$ &=& $H(S, \, P) \quad $ & --- энтальпия, \\
    $F$ &=& $F(T, \, V) \quad $ & --- свободная энергия Гельмгольца,  \\
    $G$ &=& $G(T, \, P) \quad $ & --- потенциал Гиббса,
\end{tabular}
\end{center}
здесь $V$ --- объем, $T$ --- температура, $P$ --- давление.
Имея выражение для одного из потенциалов в естественных переменных, можно получить выражение для любого другого потенциала в любых переменных, а также зависимость любой переменной от пары других.

К примеру, пусть задано каноническое уравнение состояния $U(S, \, V)$, тогда из первого начала термодинамики следуют выражения для температуры и давления
\begin{equation*}
    dU = T \, dS - P \, dV \quad \Rightarrow \quad 
    T(S, \, V) = \qty(\pdv{U}{S})_{V},
    \quad
    P(S, \, V) = -\qty(\pdv{U}{V})_{S},
\end{equation*}
здесь индексы $_{V}$ и $_{S}$ при производных обозначают дифференцирование при постоянном объеме и энтропии соответственно.

В расчетной практике удобнее использовать удельные (отнесенные к массе) величины: удельный объем $\upsilon = 1 / \rho$, где $\rho$ --- плотность, удельная внутренняя энергия $e$, удельная энтропия $s$, иногда также удельная энтальпия $h$. В дальнейшем слово <<удельный>> в большинстве случаев будет опускаться. Величины $e$ и $h$ имеют размерность квадрата скорости.
\smallskip

Пусть у нас имеется <<практически каноническое>> уравнение состояния для удельной внутренней энергии $e = e(\rho, \, s)$. Для удельной внутренней энергии также справеливо первое начало термодинамики, которое можно записать в виде
\begin{equation}\label{eq:first_law}
    d e = T \, ds + \frac{P}{\rho^2} \, d\rho,
\end{equation}
отсюда можно получить формулы для давления и температуры в естественных переменных
\begin{equation}\label{eq:P_and_T}
    P(\rho, \, s) = \rho^2 \, \qty(\pdv{e}{\rho})_{s} 
    \qquad \text{и} \qquad
    T(\rho, \, s) = \qty(\pdv{e}{s})_{\rho}.
\end{equation}
Энтропию можно исключить из формул, для этого достаточно выразить $s \qty(\rho, \, e)$ из выражения для внутренней энергии и подставить в формулы \eqref{eq:P_and_T}. Таким образом, можно считать, что у нас имеются выражения $P \qty(\rho, \, e)$ и $T \qty(\rho, \, e)$.

Другим важным параметром для газодинамики является скорость звука $c$. Квадрат скорости звука определяется в изоэнтропийном процессе
\begin{equation}\label{eq:sound_speed_s}
    c^2(\rho, \, s) = \qty(\pdv{P}{\rho})_{s},
\end{equation}
Поскольку в расчетах чаще используется зависимость без энтропии $P(\rho, \, e)$, выражение для скорости звука будет удобнее переписать в переменных $(\rho, \, e)$.
Рассмотрим изоэнтропийный процесс, из первого начала \eqref{eq:first_law} следует, что вся внутренняя \textit{энергия идет на работу} $de = P \, d\rho / \rho^2$, запишем полный дифференциал давления в различных переменных:
\begin{align*}
    & dP(\rho, \, s) = \qty(\pdv{P}{\rho})_{s} \, d\rho = c^2 \, d \rho \\
    & dP(\rho, \, e) = \qty(\pdv{P}{\rho})_{e} \, d \rho +
    \qty(\pdv{P}{e})_{\rho} \, d e =
    \qty[ \qty(\pdv{P}{\rho})_{e} + \frac{P}{\rho^2} \, \qty(\pdv{P}{e})_{\rho} ] \, d \rho
\end{align*}
откуда следует, что
\begin{equation}\label{eq:sound_speed_re}
    c^2(\rho, \, e) = \qty(\pdv{P}{\rho})_{e} + \frac{P}{\rho^2} \, \qty(\pdv{P}{e})_{\rho}.
\end{equation}
\medskip

Использование энтропии на практике не очень удобно, поэтому вместо канонического уравнения состояния обычно используется пара из термического уравнения состояния и калорического уравнения состояния. \textit{Термическое уравнение состояния} связывает  температуру, давление и объем (плотность). \textit{Калорическое уравнение состояния} включает зависимость для внутренней энергии. Рассмотрим для примера идеальный газ. Термическое уравнение состояния для идеального газа это уравнение Менделеева -- Клапейрона:
\begin{equation*}
    P = \frac{R}{\mu} \rho T = \qty(\gamma - 1) c_v \rho \, T,
\end{equation*}
в формуле учтено, что $R = (\gamma - 1) \mu c_v$, $\gamma$ --- показатель адиабаты, $R$ --- универсальная газовая постоянная, $\mu$ --- молярная масса. Внутренняя энергия идеального газа не зависит от объема и давления, калорическое уравнение состояния имеет вид $e = c_v T$, где $c_v$ --- удельная изохорная теплоемкость.

Классический вид законов, используемый в газодинамике для описания идеальных газов:
\begin{equation*}
    P(\rho, \, e) = \qty(\gamma - 1) \rho e, \qquad \qquad
    e(\rho, \, T) = c_v T.
\end{equation*}

Каноническое уравнение состояния $e(\rho, \, s)$ исчерпывающе описывает термодинамическую систему, в то время как только одного из уравнений $P(\rho, \, e)$ или $e( \rho, \, T)$ недостаточно для полного описания. 

Наличие двух зависимостей позволяет восстановить вид канонического уравнения состояния $e(\rho, \, s)$. Выполним на примере идеального газа. Как и ранее, пишем полный дифференциал внутренней энергии (= первое начало термодинамики):
\begin{multline*}
    de = 
    T(\rho, \, e(\rho, \, s)) ds + 
        \frac{P (\rho, \, e(\rho, \, s)) }{\rho^2} \, d\rho =
    \frac{1}{c_v} \, e(\rho, s) \, ds + 
        \frac{1}{\rho^2} (\gamma - 1) \, \rho \, e(\rho, \, s) \, d \rho = \\
    e(\rho, \, s) \qty( d \frac{s}{c_v} + \qty(\gamma - 1) \, d \ln\rho ) =
    e(\rho, \, s) \, d \qty(\frac{s}{c_v} + \ln \rho^{\gamma - 1} ).
\end{multline*}
Далее разделяем переменные:
\begin{equation*}
    \frac{de}{e(\rho, \, s)} = d \ln e =
     d \qty(\frac{s}{c_v} + \ln \rho^{\gamma - 1} ),
\end{equation*}
после интегрирования получаем:
\begin{equation*}
    e(\rho, \, s) = e_0 \exp \qty(\tfrac{s - s_0}{c_v}) \qty(\frac{\rho}{\rho_0})^{\gamma - 1}
\end{equation*}
--- это и есть каноническое уравнение состояния идеального газа. Физики-теоретики ту же формулу запишут в виде
\begin{equation}
    U(S, \, V, \, N) = \hat{c}_v k N \qty( \frac{N \Phi}{V} \exp\qty(\frac{S}{k N}) )^{\frac{1}{\hat{c}_v}},
\end{equation}
где $\hat{c}_V$ --- безразмерная теплоемкость при постоянном объеме, $N$ --- число частиц, $k$ --- постоянная Больцмана и $\Phi$ --- некоторая константа.

\section{Некоторые уравнения состояния}

\newcommand{\eqindent}{3\parindent}


Несколько слов о том, какие зависимости представляются и почему они.


\addcontentsline{toc}{subsection}{Идеальный газ}
\subsection*{Идеальный газ}

Простейшее уравнение состояния:
\begin{fleqn}[\eqindent]
\allowdisplaybreaks
\begin{align*}
& P(\rho, \, e) = (\gamma - 1) \rho e, \\
& P(\rho, \, T) = (\gamma - 1) c_v \rho T, \\
& e(\rho, \, P) = \frac{P}{(\gamma - 1) \rho}, \\
& e(\rho, \, T) = c_v T, \\
& c^2 \qty(\rho, \, P) = \gamma \frac{P}{\rho}, \\
& c^2 \qty(\rho, \, e) = \gamma \qty(\gamma - 1) e, \\
& c^2 \qty(\rho, \, T) = \gamma \qty(\gamma - 1) c_v T.
\end{align*}
\end{fleqn}

Каноническое уравнение состояния:
\begin{fleqn}[\eqindent]
\begin{equation*}
e \qty( \rho, \, s ) = A \exp \qty(\tfrac{s - s_0}{c_v}) \qty(\frac{\rho}{\rho_0})^{\gamma - 1}.
\end{equation*}
\end{fleqn}

\addcontentsline{toc}{subsection}{Двучленное уравнение состояния}
\subsection*{Двучленное уравнение состояния}

Двучленное уравнение состояния является несложным обобщением уравнения состояния идеального газа. Двучленное уравнение состояния используется для моделирования процессов в жидкостях, а также в металлах при высоких давлениях. К формуле для давления добавляется линейное слагаемое по плотности \cite{Godunov76}:
\begin{equation*}
P(\rho, \, e) = (\gamma - 1) \rho e + c_0^2 (\rho - \rho_0).
\end{equation*}
Вместо пары констант $c_0$, $\rho_0$ часто используются константы:
\begin{equation*}
P_0 = \frac{\rho_0 c_0^2}{\gamma}, \qquad e_0 = - \frac{c_0^2}{\gamma - 1}.
\end{equation*}
С такими константами уравнение принимает вид:
\begin{equation*}
P(\rho, \, e) = (\gamma - 1) \rho (e - e_0) - \gamma P_0.
\end{equation*}
Хотя такая замена не эквивалентна. Первая формула указывает на положительность коэффициента перед плотностью ($c_0^2$), а при $c_0 = 0$ автоматически исключается параметр $\rho_0$. Двучленное уравнение состояния переходит в уравнение состояния идеального газа при $P_0 = 0$, $e_0 = 0$ или $c_0 = 0$.

Единственного уравнения $P(\rho, \, e)$ достаточно при решении задач механики сплошной среды без участия температуры. Однако для полного описания термодинамической системы его следует дополнить термическим уравнением состояния. Термическое уравнение состояния можно вывести из $P(\rho, \, e)$, если сделать допущение $\qty( de / dT )_{\rho} = c_v = \const$. Для этого необходимо воспользоваться общей теорией термодинамики, покрутить термодинамические потенциалы и соотношения Максвелла. Пример такого вывода есть в работе Сауреля \cite{Saurel16} для уравнения состояния, близкого к двучленному.

Допущение $c_v = \const$ означает, что $e(\rho, \, T) = c_v T + f(\rho)$. После различных преобразований (выкладки есть в разделе \ref{app:stiffened}), решения дифференциальных уравнений, мы приходим к тому, что энтропийная функция $\sigma(s)$ определяется с точностью до двух констант $e_1$, $e_2$:
\begin{equation*}
\sigma(s) = e_1 \exp \qty( \tfrac{s - s_0}{c_v} ) + e_2.
\end{equation*}
При этом константа $e_1$ не влияет на форму термического уравнения состояния, а от выбора константы $e_2$ зависит выражение $P(\rho, \, T)$. 

Первый вариант --- просто положить $e_2 = 0$. Так и поступает Саурель, в упомянутой выше работе \cite{Saurel16}. Если честно, его объяснение, почему он так делает, не кажется очень убедительным. При таком подходе получается следующее уравнение состояния:
\begin{fleqn}[\eqindent]
\allowdisplaybreaks
\begin{align*}
& P(\rho, \, e) = (\gamma - 1) \rho (e - e_0) - \gamma P_0, \\
& P(\rho, \, T) = (\gamma - 1) c_v \rho T - P_0, \\
& e(\rho, \, P) = \frac{P + \gamma P_0}{(\gamma - 1) \rho}, \\
& e(\rho, \, T) = c_v T + e_0 + \frac{P_0}{\rho}, \\
& c^2 \qty(\rho, \, e) = \gamma \qty(\gamma - 1) \qty( e - e_0 - \frac{P_0}{\rho} ), \\
& c^2 \qty(\rho, \, P) = \gamma \frac{P + P_0}{\rho}, \\
& c^2 \qty(\rho, \, T) = \gamma \qty(\gamma - 1) c_v T.
\end{align*}
\end{fleqn}

Каноническое уравнение состояния:
\begin{fleqn}[\eqindent]
\begin{equation*}
e \qty( \rho, \, s ) = A \exp \qty(\tfrac{s - s_0}{c_v}) \qty(\frac{\rho}{\rho_0})^{\gamma - 1} + \frac{P_0}{\rho} + e_0.
\end{equation*}
\end{fleqn}

Уравнение состояния в таком виде чаще используется в англоязычной литературе, оно встречается под названием \textbf{<<Stiffened Gas>>} (жёсткий газ?).
\medskip

Если константа $e_2$ отлична от нуля, тогда получаем формулы:
\begin{fleqn}[\eqindent]
\allowdisplaybreaks
\begin{align*}
& P(\rho, \, e) = (\gamma - 1) \rho (e - e_0) - \gamma P_0, \\
& P(\rho, \, T) = (\gamma - 1) c_v \rho T + P_0 \qty[\qty(\frac{\rho}{\rho_0})^{\gamma} - 1  ], \\
& e(\rho, \, P) = \frac{P + \gamma P_0}{(\gamma - 1) \rho}, \\
& e(\rho, \, T) = c_v T + e_0 + \frac{P_0}{\rho} - \frac{e_0}{\gamma} \qty(\frac{\rho}{\rho_0})^{\gamma - 1}, \\
& c^2 \qty(\rho, \, P) = \gamma \frac{P + P_0}{\rho}, \\
& c^2 \qty(\rho, \, e) = \gamma \qty(\gamma - 1) \qty( e - e_0 - \frac{P_0}{\rho} ), \\
& c^2 \qty(\rho, \, T) = \gamma \qty(\gamma - 1) c_v T + c_0^2 \qty( \frac{\rho}{\rho_0})^{\gamma - 1}.
\end{align*}
\end{fleqn}

Каноническое уравнение состояния:
\begin{fleqn}[\eqindent]
\begin{equation*}
e \qty( \rho, \, s ) = \qty[ A \exp \qty(\tfrac{s - s_0}{c_v}) - \frac{e_0}{\gamma}] \qty(\frac{\rho}{\rho_0})^{\gamma - 1} + \frac{P_0}{\rho} + e_0.
\end{equation*}
\end{fleqn}

Константа $e_2$ подобрана таким образом, что упругая часть давления в выражении $P(\rho, \, T)$ обращается в ноль при $\rho = \rho_0$ \cite{Shurshalov20}. Данная форма уравнений состояния отличается от Stiffened Gas наличием степенного слагаемого в выражениях с температурой, а также каноническим уравнением состояния. Годунов в работе \cite{Godunov76} не конкретизирует вид энтропийной функции, поскольку в расчетах газодинамики это не имеет значения. Ключевое отличие двух форм двучленного уравнения состояния проявляется именно в канонической форме и обусловлено выбором энтропийной функции.

Уравнение состояния в таком виде чаще встречается в русскоязычной литературе, к примеру, у Шуршалова \cite{Shurshalov20} и Хищенко \cite{Khischenko17}, а Хищенко мы доверяем. Аналогичная форма двучленного уравнения состояния приводится в работе Лемонса \cite{Lemons99}. В этой работе уравнение выводится из общих соображений для уравнения состояния в форме Ми -- Грюнайзена. При разложении произвольной функции по малому параметру $(\rho - \rho_0)$ получается аналогичное выражение для энергии $e(\rho, \, T)$.
\smallskip

Двучленное уравнение состояния допускает отрицательные давления (разрежение), минимальное давление составляет $-P_0$.
  
  
\addcontentsline{toc}{subsection}{Уравнение состояния Ми -- Грюнайзена}
\subsection*{Уравнение состояния Ми -- Грюнайзена}

Аналогичное уравнение состояние было реализовано в классе \class{Monaghan} и в классе \class{Linear} (с параметром $n = 1$):
\begin{equation}
P_{ref} \qty(\ups) = \frac{B}{n} \qty[ \qty(\frac{\ups_0}{\ups})^{n} - 1],
\end{equation}
\begin{equation}
e_{ref} \qty(\ups) = \frac{B}{n \qty(n - 1)} \qty( \qty[ \qty(\frac{\ups_0}{\ups})^{n} + n - 1] \ups - n \ups_0 ),
\end{equation}
здесь референсной является кривая холодного давления $P_{ref} (\ups)$, при этом референсная энергия получается путем интегрирования, то есть референсные давление и энергия связаны формулами:
\begin{equation}
    e'_{ref} \qty( \ups ) = - P_{ref} \qty( \ups ).
\end{equation}
Чтобы быстро считать производные
\begin{equation}
    P'_{ref} \qty( \ups ) = - \frac{n P_{ref} \qty( \ups ) + B}{\ups}.
\end{equation}

Основная формула
\begin{equation}
P \qty(\ups, \, e ) - P_{ref} \qty( \ups ) = \frac{\Gamma}{\ups} \qty( e - e_{ref} \qty( \rho ) ). 
\end{equation}
Производные, без необходимости больших повторных вычислений
\begin{equation}
\qty( \pdv{P}{\ups} )_{e} = - \frac{1}{\ups} (P + B + (n - \Gamma - 1) \cdot P_{ref} (\ups) ), \qquad
\qty( \pdv{P}{e} )_{\ups} = \frac{\Gamma}{\ups}.
\end{equation}
Отсюда можно выразить скорость звука
\begin{equation}
c^2 \qty(\ups, \, P) = \ups \qty( B + \qty( 1 + \Gamma ) P + \qty( n - 1 - \Gamma ) P_{ref} \qty( \ups ) )
\end{equation}
Если скомбинировать $P$ и $P_{ref}$, то можно получить зависимость в другой паре переменных:
\begin{equation}
c^2(\ups, e) = \Gamma \qty( \Gamma + 1 ) \qty(e - e_{ref} \qty(\ups) ) + \ups \qty(B + \gamma P_{ref} \qty(\ups) ).
\end{equation}

Удельный объем $\ups \qty(P, \, T)$ удовлетворяет нелинейному уравнению
\begin{equation}
\ups \qty( P - P_{ref}(\ups) ) = \Gamma C_v \qty( T - T_0 ),
\end{equation}
Уравнение имеет следующий вид:
\begin{equation}
    A x - x^{-\nu} = C, \qquad \ups = \ups_0 x,
\end{equation}
константы
\begin{equation}
\nu = \gamma - 1, \qquad A = 1 + \frac{n P}{B}, \qquad C = \frac{n \Gamma C_v }{\ups_0 B} \qty(T - T_0),
\end{equation}
при положительных $A$ уравнение имеет единственное решение, при $A \le 0$ решение может не существовать или не быть единственным. Условие эквивалентно следующему: 
\begin{equation}
    P > P_{\min} = - B / n.
\end{equation}
Производные $\ups$ выражаются следующим образом:
\begin{equation}
    \qty(\pdv{\ups}{P})_{T} = - \frac{\ups}{D}, \qquad
    \qty(\pdv{\ups}{T})_{P} = \frac{\Gamma C_v}{D}, \qquad
    D = B + P + \qty(n - 1) P_{ref} \qty(\ups).
\end{equation}

Энергия от $(P, \, T)$ определяется после нахождения удельного объема, а производные:
\begin{equation}
    \qty(\pdv{e}{P})_{T} = - P_{ref} \qty( \ups ) \qty( \pdv{\ups}{P} )_T, \qquad
    \qty(\pdv{e}{P})_{T} = C_v - P_{ref}c \qty( \ups ) \qty( \pdv{\ups}{T} )_P.
\end{equation}
\bigskip


\begin{equation}
P_{c} \qty(\ups) = \frac{B}{\gamma} \qty[ \qty(\frac{\ups_0}{\ups})^{\gamma} - 1], \qquad
e_{c} \qty(\ups) = \frac{B}{\gamma \qty(\gamma - 1)} \qty( \qty[ \qty(\frac{\ups_0}{\ups})^{\gamma} + \gamma - 1] \ups - \gamma \ups_0 ),
\end{equation}
\begin{equation}
P_{c} \qty(\rho) = \frac{B}{\gamma} \qty[ \qty(\frac{\rho}{\rho_0})^{\gamma} - 1], \qquad
e_{c} \qty(\rho) = \frac{B}{\gamma \qty(\gamma - 1)} \qty( \qty[ \qty(\frac{\rho}{\rho_0})^{\gamma} + \gamma - 1 ] \frac{1}{\rho} - \gamma \frac{1}{\rho_0} ),
\end{equation}

Справедливы соотношения
\begin{equation}
    e'_{c} \qty( \ups ) = - P_{c} \qty( \ups ),
    \qquad
    e'_{c} \qty( \rho ) = \frac{1}{\rho^2} P_{c} \qty( \rho ),
\end{equation}
Чтобы быстро считать производные
\begin{equation}
    P'_{c} \qty( \ups ) = - \frac{\gamma P_{c} \qty( \ups ) + B}{\ups}, \qquad
    P'_{c} \qty( \rho ) = \frac{\gamma P_{c} \qty( \rho ) + B}{\rho}.
\end{equation}

Основная формула
\begin{equation}
P \qty(\rho, \, e ) - P_{c} \qty( \rho ) = \Gamma \rho \qty( e - e_{c} \qty( \rho ) ). 
\end{equation}
Производные, без необходимости больших повторных вычислений
\begin{equation}
\qty( \pdv{P}{\rho} )_{e} = \frac{1}{\rho} \qty( P + B + \qty( \gamma - 1 - \Gamma ) P_{c} \qty( \rho ) ), \qquad
\qty( \pdv{P}{e} )_{\rho} = \Gamma \rho.
\end{equation}
Отсюда можно выразить скорость звука
\begin{equation}
c^2 \qty(\rho, \, P) = \frac{1}{\rho} \qty( B + \qty( 1 + \Gamma ) P + \qty( \gamma - 1 - \Gamma ) P_{c} \qty( \rho ) )
\end{equation}
Если скомбинировать $P$ и $P_c$, то можно получить зависимость в другой паре переменных:
\begin{equation}
c^2(\rho, e) = \Gamma \qty( \Gamma + 1 ) \qty(e - e_c \qty(\rho ) ) + \frac{B + \gamma P_c \qty(\rho)}{\rho}.
\end{equation}

Удельный объем $\ups \qty(P, \, T)$ удовлетворяет нелинейному уравнению
\begin{equation}
\ups \qty( P - P_c(\ups) ) = \Gamma C_v \qty( T - T_0 ),
\end{equation}
Уравнение имеет следующий вид:
\begin{equation}
    A x - x^{-\nu} = C, \qquad \ups = \ups_0 x,
\end{equation}
константы
\begin{equation}
\nu = \gamma - 1, \qquad A = 1 + \frac{\gamma P}{B}, \qquad C = \frac{\gamma \Gamma C_v }{\ups_0 B} \qty(T - T_0),
\end{equation}
при положительных $A$ уравнение имеет единственное решение, при $A \le 0$ решение может не существовать или не быть единственным. Условие эквивалентно следующему: 
\begin{equation}
    P > P_{\min} = - B / \gamma.
\end{equation}
Производные $\ups$ выражаются следующим образом:
\begin{equation}
    \qty(\pdv{\ups}{P})_{T} = - \frac{\ups}{D}, \qquad
    \qty(\pdv{\ups}{T})_{P} = \frac{\Gamma C_v}{D}, \qquad
    D = B + P + \qty(\gamma - 1) P_c \qty(\ups).
\end{equation}

Энергия от $(P, \, T)$ определяется после нахождения удельного объема, а производные:
\begin{equation}
    \qty(\pdv{e}{P})_{T} = - P_c \qty( \ups ) \qty( \pdv{\ups}{P} )_T, \qquad
    \qty(\pdv{e}{P})_{T} = C_v - P_c \qty( \ups ) \qty( \pdv{\ups}{T} )_P.
\end{equation}

Остались формулы для StiffenedGas.

Есть подозрение, что следует делать вот так
\begin{equation}
T_{ref}(\rho) = T_0 \qty(\frac{\rho}{\rho_0} )^{\Gamma},
\end{equation}
\begin{equation}
e(\rho, \, T) = e_{ref} (\rho) + c_v \qty( T - T_{ref}(\rho) ).
\end{equation}
\begin{equation}
P(\rho, \, T) = P_{ref} (\rho) + \Gamma c_v \rho \qty( T - T_{ref}(\rho) ).
\end{equation}

И каноническое уравнение состояния
\begin{equation}
e(\rho, \, s) = \qty[A \exp \qty( \tfrac{s - s_0}{c_v} ) - c_v T_0] \qty( \frac{\rho}{\rho_0})^{\Gamma} + e_{ref} (\rho).
\end{equation}


\addcontentsline{toc}{subsection}{Уравнение состояния Гюгонио}
\subsection*{Уравнение состояния Гюгонио}

Референсная кривая $P_{ref}(\ups)$ задается в виде адиабаты Гюгонио, где $\xi = \ups / \ups_0$:
\begin{equation}\label{eq:hugoniot-b}
P_{ref}( \xi ) = \left\{
\begin{aligned}
& \frac{B \cdot (1 - \xi)}{(1 - a(1 - \xi))^2}, \; &\xi < 1, \quad 
&\text{-- <<сжатие>>}, \\
& B \cdot \qty(1 - \xi), \; &\xi \ge 1, \quad 
&\text{-- <<разгрузка>>}.
\end{aligned}
\right.
\end{equation}
при этом референсная кривая для энергии $e_{ref}(\ups)$ также должна быть задана в соответствии с ударной адиабатой:
\begin{equation}
e_{ref}(\ups) = \half \, P_{ref}(\ups) \qty( \ups_0 - \ups ).
\end{equation}

Здесь $B$ --- объемный модуль упругости (bulk modulus) при референсном удельном объеме $\ups_0$. Энергия и давление связаны по закону Ми--Грюнайзена:
\begin{equation}
\ups \cdot \qty( P - P_{ref}( \ups ) ) \;=\; \Gamma \cdot \qty( e - e_{ref}( \ups ) ).
\end{equation}
\smallskip

\textbf{Процедура Уолша--Кристиана.}
\smallskip

Пусть референсная кривая $P( \ups )$ --- адиабата Гюгонио, тогда референсная температурная кривая $T( \ups )$ является решением следующего обыкновенного дифференциального уравнения \cite{Walsh55}:
\begin{equation}\label{eq:walsh}
T'(\ups) + \frac{\Gamma}{V_0} T( \ups ) = \frac{1}{C_v} f( \ups ) = \frac{1}{2 C_v} ( P'( \ups ) (\ups_0 - \ups) + P( \ups ) ).
\end{equation}
  
Если перейти к безрамерной величине $\xi = \ups / \ups_0$, то получим задачу Коши  
\begin{equation}
T'(\xi) + \Gamma \, T(\xi) = \frac{V_0}{2C_v} ( P'(\xi) (1 - \xi) + P(\xi) ), \quad T(1) = T_0, \quad \xi = \ups / \ups_0,
\end{equation}  
решение которой выражается в виде интеграла.

Применение процедуры Уолша--Кристиана для кривой \eqref{eq:hugoniot-b} при разгрузке $(\xi \ge 1)$ дает простое выражение для $T_{ref}(\ups)$:
\begin{equation}
T_{ref}( \xi ) = T_0 \cdot e^{\Gamma \qty(1 - \xi )}, \quad \xi \ge 1,
\end{equation}

При сжатии ($\xi < 1$) формула несколько усложняется:
\begin{equation*}
    T_{ref}( \xi ) = 
    \left(
        T_0 + \frac{B \ups_0}{2 a^2 C_v}
        \Big[
            3 - g + A \, e^{-\Gamma (1 - \xi)} - 
            C \big( \mathrm{Ei}( g X ) - \mathrm{Ei}( g ) \big)
        \Big]
    \right) 
    e^{\Gamma (1 - \xi)}
\end{equation*}
\begin{equation*}
    g = \frac{\Gamma}{a}, \quad
    X = 1 - a (1 - \xi), \quad
    A = \frac{(4a - \Gamma)(1 - \xi) - 3 + g}{X^2}, \quad
    C = \left( 2 - 4 g + g^2 \right) e^{-g}
\end{equation*}
\medskip

Энергия и температура связаны по формуле, похожей на закон Ми--Грюнайзена:
\begin{equation}
e - e_{ref}(\ups) \;=\; C_v \cdot \qty( T - T_{ref}(\ups) ).
\end{equation}

\addcontentsline{toc}{subsection}{Уравнение состояния Тейта}
\subsection*{Уравнение состояния Тейта}

Референсная кривая $P_{ref}(\ups)$ задается из двух частей, давление на сжатие совпадает с формулой из раздела <<Ми--Грюнайзен>>, давление при растяжении совпадает с формулой из раздела <<Гюгонио>>:
\begin{equation}\label{eq:teit}
P_{ref}( \xi ) = \left\{
\begin{aligned}
& \frac{B}{n} \qty[ \qty(\frac{\ups_0}{\ups})^{n} - 1], \; &\ups < \ups_0, \quad 
&\text{-- <<сжатие>>}, \\
& B \cdot \qty(1 - \frac{\ups}{\ups_0}), \; &\ups \ge \ups_0, \quad 
&\text{-- <<разгрузка>>}.
\end{aligned}
\right.
\end{equation}

\begin{equation}
e_{ref}( \xi ) = \left\{
\begin{aligned}
& \frac{B}{n \qty(n - 1)} \qty( \qty[ \qty(\frac{\ups_0}{\ups})^{n} + n - 1] \ups - n \ups_0 ), \; &\ups < \ups_0, \quad 
&\text{-- <<сжатие>>}, \\
& \frac{B v_0}{2} \cdot \qty(1 - \frac{\ups}{\ups_0})^2, \; &\ups \ge \ups_0, \quad 
&\text{-- <<разгрузка>>}.
\end{aligned}
\right.
\end{equation}

Референсные кривые удовлетворяют дифференциальному соотношению на всём диапазоне
\begin{equation}
    e'_{ref} \qty( \ups ) = - P_{ref} \qty( \ups ),
\end{equation}
кроме того, на участке разгрузки 
\begin{equation}
e_{ref}(\ups) = \half \, P_{ref}(\ups) \qty( \ups_0 - \ups ).
\end{equation}

Сейчас используется референсная температура в виде константы $T_{ref}(\ups) = T_0$, хотя можно было бы в области разгрузки сделать экспоненту:
\begin{equation}
T_{ref}( \xi ) = \left\{
\begin{aligned}
& T_0, \; &\ups < \ups_0, \quad 
&\text{-- <<сжатие>>}, \\
& T_0 \exp \qty[ \Gamma \qty(1 - \frac{\ups}{\ups_0}) ], \; &\ups \ge \ups_0, \quad 
&\text{-- <<разгрузка>>}.
\end{aligned}
\right.
\end{equation}

\addcontentsline{toc}{subsection}{Уравнение состояния Бёрча -- Мурнагана}
\subsection*{Уравнение состояния Бёрча -- Мурнагана}

Часто используется, будет в качестве примера жести, плюс надо разобраться.



\section{Уравнение состояния смеси}

При заданных массовых концентрациях компонент $\beta$, уравнение состояния для смеси определяется следующим образом:
\begin{equation}\label{eq:closure}
\left\{
\begin{aligned}
&\ups = \ups^{\beta} \qty(P, \, T) = \sum_i \beta_i \ups_i \qty( P, \, T), \\
&e = e^{\beta} \qty(P, \, T) = \sum_i \beta_i e_i \qty( P, \, T),
\end{aligned}
\right.
\end{equation}
здесь $\ups = 1 / \rho$ --- удельный объем, $\ups_i (P, \, T)$ и $e_i (P, \, T)$ уравнения состояния для отдельных компонент. В соответствии с уравнением \eqref{eq:closure} явным образом определяется только внутренняя энергия смеси $e \qty(P, \, T)$ и удельный объем смеси $\ups \qty(P, \, T)$, что эквивалентно заданию $\rho \qty( P, \, T)$.

Численные схемы обычно требуют получение зависимостей от других переменных. Так, система уравнений гидродинамики вообще не включает температуру, поэтому исключение температуры из системы \eqref{eq:closure} является важной процедурой. Далее перечислены фукнции, необходимые для работы с большинством численных методик.
\medskip

1. Определить $P$ при известных $\ups$ (или $\rho$) и $T$:
\begin{equation}
P^{k+1} = P^{k} + \frac{\ups - \ups^{\beta}}{\ups^{\beta}_P }.
\end{equation}

2. Определить $T$ при известных $\ups$ (или $\rho$) и $P$:
\begin{equation}
T^{k+1} = T^{k} + \frac{\ups - \ups^{\beta}}{\ups^{\beta}_T}.
\end{equation}

В схемах выше используются ньютоновские итерации по первому уравнению системы \eqref{eq:closure}.
\medskip

3. Найти $P$ и $T$ при известных $\rho$ и $e$:
\begin{equation}
\begin{aligned}
& \Delta = \ups^{\beta}_T \, e^{\beta}_P - \ups^{\beta}_P \, e^{\beta}_T, \\
& P^{k+1} = P^{k} + \frac{ \qty( e - e^{\beta} ) \, \ups^{\beta}_T - \qty(\ups - \ups^{\beta}  ) \, e^{\beta}_{T} }{\Delta} & \\
& T^{k+1} = T^{k} + \frac{\qty( \ups - \ups^{\beta} ) \, e^{\beta}_P - \qty(e - e^{\beta}  ) \, \ups^{\beta}_{P} }{\Delta}. 
\end{aligned}
\end{equation}

В данном случае используются ньютоновские итерации по обоим уравнениям системы \eqref{eq:closure}.
\medskip

В итерационных формулах выше используются следующие обозначения: $\ups^{\beta}$ и $e^{\beta}$ обозначают удельный объем и внутреннюю энергию смеси от переменных $\qty(P, \, T)$, нижним индексом обозначаются производные смесевых величин по $P$ или $T$. К примеру:
\begin{equation*}
    \ups^{\beta}_P = \qty(\pdv{\ups^{\beta}}{P})_{T} = 
    \sum_{i} \beta_i \qty(\pdv{\ups^{\beta}_i}{P})_{T}, \qquad e^{\beta}_T = \qty(\pdv{e^{\beta}}{T})_{P} = 
    \sum_{i} \beta_i \qty(\pdv{e^{\beta}_i}{T})_{P}.
\end{equation*}
\medskip

Кроме того, могут потребоваться производные:
\begin{equation}
    \qty(\pdv{p}{e})_{\ups} = \frac{\ups^{\beta}_T}{\Delta}, \qquad
    \qty(\pdv{p}{\ups})_{e} = -\frac{e^{\beta}_T}{\Delta}.
\end{equation}

Формулы получаются после дифференцирования \eqref{eq:closure}. Производные в переменных $\qty( \rho, \, e)$:
\begin{equation}
    \qty(\pdv{p}{e})_{\rho} = \qty(\pdv{p}{e})_{\ups} =  \frac{\ups^{\beta}_T}{\Delta}, \qquad
    \qty(\pdv{p}{\rho})_{e} = -\ups^2 \qty(\pdv{p}{\ups})_{e} = \ups^2 \frac{e^{\beta}_T}{\Delta}.
\end{equation}

Выражение для скорости звука:
\begin{equation}\label{eq:pt_sound}
    c^2 = \ups^2 \qty( p \qty(\pdv{p}{e})_{\ups} - \qty(\pdv{p}{\ups})_{e} ) =
    \ups^2 \, \frac{ e^{\beta}_{T} + p \ups^{\beta}_T }{\Delta}.
\end{equation}

Константы двучленного уравнения состояния:
\begin{equation}
    P\qty(\rho, e) = \qty(\gamma - 1) \rho \qty(e - e_0) - \gamma P_0.
\end{equation}
\begin{equation}
    \gamma = 1 + \ups \frac{\ups^{\beta}_T}{\Delta},
    \qquad
    e_0 = e - \ups \frac{e^{\beta}_T}{\ups^{\beta}_T},
    \qquad
    P_0 = \frac{1}{\gamma} \qty( \ups \frac{e^{\beta}_T}{\Delta} - P).
\end{equation}

Константы аппроксимации можно получить, если приравнять значения давления и первых производных при заданом состоянии $(\rho, \, e, \, P)$. Скорость звука определяется по первым производным, поэтому скорость звука, полученная по формуле \eqref{eq:pt_sound} будет сопадать со скоростью звука для двучленной аппроксимации (легко проверяется).

\subsection{Схема 1.}

При заданых массовых концентрациях компонент $\beta$, уравнение состояния для смеси определяется следующим образом:
\begin{equation}\label{eq:closure}
\left\{
\begin{aligned}
&\ups = \ups^{\beta} \qty(P, \, T) = \sum_i \beta_i \ups_i \qty( P, \, T), \\
&e = e^{\beta} \qty(P, \, T) = \sum_i \beta_i e_i \qty( P, \, T),
\end{aligned}
\right.
\end{equation}
здесь $\ups = 1 / \rho$ --- удельный объем, $\ups_i (P, \, T)$ и $e_i (P, \, T)$ уравнения состояния для отдельных компонент. В соответствии с уравнением \eqref{eq:closure} явным образом определяется только внутренняя энергия смеси $e \qty(P, \, T)$ и удельный объем смеси $\ups \qty(P, \, T)$, что эквивалентно заданию $\rho \qty( P, \, T)$.

Численные схемы обычно требуют получение зависимостей от других переменных. Так, система уравнений гидродинамики вообще не включает температуру, поэтому исключение температуры из системы \eqref{eq:closure} является важной процедурой. Далее перечислены фукнции, необходимые для работы с большинством численных методик.
\medskip

1. Определить $P$ при известных $\ups$ (или $\rho$) и $T$:
\begin{equation}
P^{k+1} = P^{k} + \frac{\ups - \ups^{\beta}}{\ups^{\beta}_P }.
\end{equation}

2. Определить $T$ при известных $\ups$ (или $\rho$) и $P$:
\begin{equation}
T^{k+1} = T^{k} + \frac{\ups - \ups^{\beta}}{\ups^{\beta}_T}.
\end{equation}

В схемах выше используются ньютоновские итерации по первому уравнению системы \eqref{eq:closure}.
\medskip

3. Найти $P$ и $T$ при известных $\rho$ и $e$:
\begin{equation}
\begin{aligned}
& \Delta = \ups^{\beta}_T \, e^{\beta}_P - \ups^{\beta}_P \, e^{\beta}_T, \\
& P^{k+1} = P^{k} + \frac{ \qty( e - e^{\beta} ) \, \ups^{\beta}_T - \qty(\ups - \ups^{\beta}  ) \, e^{\beta}_{T} }{\Delta} & \\
& T^{k+1} = T^{k} + \frac{\qty( \ups - \ups^{\beta} ) \, e^{\beta}_P - \qty(e - e^{\beta}  ) \, \ups^{\beta}_{P} }{\Delta}. 
\end{aligned}
\end{equation}

В данном случае используются ньютоновские итерации по обоим уравнениям системы \eqref{eq:closure}.
\medskip

В итерационных формулах выше используются следующие обозначения: $\ups^{\beta}$ и $e^{\beta}$ обозначают удельный объем и внутреннюю энергию смеси от переменных $\qty(P, \, T)$, нижним индексом обозначаются производные смесевых величин по $P$ или $T$. К примеру:
\begin{equation*}
    \ups^{\beta}_P = \qty(\pdv{\ups^{\beta}}{P})_{T} = 
    \sum_{i} \beta_i \qty(\pdv{\ups^{\beta}_i}{P})_{T}, \qquad e^{\beta}_T = \qty(\pdv{e^{\beta}}{T})_{P} = 
    \sum_{i} \beta_i \qty(\pdv{e^{\beta}_i}{T})_{P}.
\end{equation*}
\medskip

Кроме того, могут потребоваться производные:
\begin{equation}
    \qty(\pdv{p}{e})_{\ups} = \frac{\ups^{\beta}_T}{\Delta}, \qquad
    \qty(\pdv{p}{\ups})_{e} = -\frac{e^{\beta}_T}{\Delta}.
\end{equation}

Формулы получаются после дифференцирования \eqref{eq:closure}. Производные в переменных $\qty( \rho, \, e)$:
\begin{equation}
    \qty(\pdv{p}{e})_{\rho} = \qty(\pdv{p}{e})_{\ups} =  \frac{\ups^{\beta}_T}{\Delta}, \qquad
    \qty(\pdv{p}{\rho})_{e} = -\ups^2 \qty(\pdv{p}{\ups})_{e} = \ups^2 \frac{e^{\beta}_T}{\Delta}.
\end{equation}

Выражение для скорости звука:
\begin{equation}\label{eq:pt_sound}
    c^2 = \ups^2 \qty( p \qty(\pdv{p}{e})_{\ups} - \qty(\pdv{p}{\ups})_{e} ) =
    \ups^2 \, \frac{ e^{\beta}_{T} + p \ups^{\beta}_T }{\Delta}.
\end{equation}

Константы двучленного уравнения состояния:
\begin{equation}
    P\qty(\rho, e) = \qty(\gamma - 1) \rho \qty(e - e_0) - \gamma P_0.
\end{equation}
\begin{equation}
    \gamma = 1 + \ups \frac{\ups^{\beta}_T}{\Delta},
    \qquad
    e_0 = e - \ups \frac{e^{\beta}_T}{\ups^{\beta}_T},
    \qquad
    P_0 = \frac{1}{\gamma} \qty( \ups \frac{e^{\beta}_T}{\Delta} - P).
\end{equation}

Константы аппроксимации можно получить, если приравнять значения давления и первых производных при заданом состоянии $(\rho, \, e, \, P)$. Скорость звука определяется по первым производным, поэтому скорость звука, полученная по формуле \eqref{eq:pt_sound} будет сопадать со скоростью звука для двучленной аппроксимации (легко проверяется).


\subsection{Схема 2.}


\subsubsection*{Начальное приближение}

Если нет начального предположения. Если появилось вещество.

\begin{equation}
\alpha^{0}_i = \frac{\beta_i \ups_i^{0}}{\sum_i \beta_i \ups^{0}_i}.
\end{equation}


\subsubsection*{Алгоритм $P(\rho, \, T)$}

Пусть известна равновесная температура $T$, плотность смеси $\rho$, а также массовые концентрации компонент $\beta_i$. Требуется определить объемные доли компонент $\alpha_i$ и найти равновесное давление $P$. Решается следующая система уравнений:
\begin{equation}
\forall i, j \quad 
P^{i} \qty( \rho_i, \, T ) = P^{j} \qty( \rho_j, \, T), 
\qquad
\sum_{i} \alpha_i = 1,
\end{equation}
где $\rho_i = \beta_i \rho / \alpha_i$. Это система на $\alpha_i$, давления здесь вообще нет. Число независимых уравнений равно $m$ --- числу компонент. Если система уравнений имеет единственное решение, тогда каждое уравнение состояния $P^i(\rho_i, \, T)$ в результате выдаст одинаковое давление. Поскольку $\beta_i$ и $\rho$ известны, можно считать, что заданы функции $P^{i} \qty(\alpha_i, \, T) = P^{i}(\beta_i \rho / \alpha_i, \, T)$. Так систему и перепишем:
\begin{equation}
\forall i, j \quad 
P^{i} \qty( \tfrac{\beta_i \rho}{\alpha_i}, \, T ) = P^{k} \qty( \tfrac{\beta_k \rho}{\alpha_k}, \, T), 
\qquad
\sum_{i} \alpha_i = 1,
\end{equation}
теперь отчетливо видно, что это система из $m$ уравнений на параметры $\alpha_i$. Будем решать систему методом Ньютона. Введем обозначения для частных производных:
\begin{equation}
P^{i}_{\alpha} = \pdv{P^{i}}{\alpha_i} = - \frac{\rho_i}{\alpha_i} \qty(\pdv{P^{i}}{\rho})_{T} = -\frac{\rho_i}{\alpha_i} P^{i}_{\rho}.
\end{equation}

Записываем метод Ньютона $\alpha_i \to \alpha_i + \Delta \alpha_i$. Пусть у нас имеется начальное приближение $\alpha^{0}_i$ такое, что уравнение $\sum_i \alpha^{0}_{i} = 1$ выполнено, тогда приращения на итерациях $\Delta \alpha_i$ должны удовлетворять условию $\sum_i \Delta \alpha_i = 0$. Другие уравнения:
\begin{equation}
P^{i} + P^{i}_{\alpha} \Delta \alpha_i = P^{j} + P^{j}_{\alpha} \Delta \alpha_j.
\end{equation}
Выразим отсюда $\Delta \alpha_i$, введем обозначение $\xi_i = - 1 / P^{i}_{\alpha}$:
\begin{equation}
\Delta \alpha_i = \xi_i \qty[P^{i} - \qty(P^{j} + P^{j}_{\alpha} \Delta \alpha_j ) ].
\end{equation}
Если посчитать сумму по $i$, то слева получится ноль, следовательно:
\begin{equation}
\sum_i \xi_i P^{i} - \sum_i \xi_i \cdot \qty[ P^{j} + P^{j}_{\alpha} \Delta \alpha_j] = 0,
\end{equation}
Введем обозначения $A_{\xi} = \sum_i \xi_i$, $B_{\xi} = \sum_i \xi_i P^{i}$, получится расчетная схема:
\begin{equation}
\Delta \alpha_i = \xi_i \cdot \qty( P^{i} \qty(\rho_i, \, T) - \frac{B_{\xi}}{A_{\xi}} ).
\end{equation}
Можно убедиться, что $\sum_i \Delta \alpha_i = 0$. 

Подробная схема вычисления. На $k$-ой итерации имеются значения $\alpha^{k}_{i}$, при этом выполнено условие $\sum_i \alpha^{k}_{i} = 1$.
\begin{equation}
\rho_i = \frac{\beta_i \rho}{\alpha^{k}_i},
\quad
\xi_i = \frac{\alpha_i}{\rho_i P_{\rho}^{i} \qty( \rho_i, \, T)},
\quad
A_{\xi} = \sum_i \xi_i,
\quad
B_{\xi} = \sum_i \xi_i P^{i}  \qty( \rho_i, \, T).
\end{equation}
\begin{equation}
\Delta \alpha_i = \xi_i \cdot \qty( P^{i} \qty(\rho_i, \, T) - \frac{B_{\xi}}{A_{\xi}} ).
\end{equation}
Затем мы $\Delta \alpha_i$ умножаем на некоторую константу $\chi$, такую, чтобы $\forall i \quad 0 < \alpha^{k}_i + \Delta \alpha_i < 1$. Подробнее расписать? После этого давление считается как среднее для всех компонент, при этом каждая компонента по факту одно и то же должна давать.

\subsubsection*{Алгоритм $P(\rho, \, e)$ и $T(\rho, \, e)$}

Пусть известны массовые концентрации компонент $\beta_i$, плотность смеси $\rho$ и внутренняя энергия смеси $e$. Требуется определить объемные доли компонент $\alpha_i$ и найти равновесные температуру $T$ и давление $P$. Будем решать следующую систему уравнений:
\begin{equation}
\forall i, j \quad 
P^{i} \qty( \rho_i, \, T ) = P^{j} \qty( \rho_j, \, T), 
\qquad
\sum_{i} \alpha_i = 1,
\qquad
\sum_{i} \beta_i e^{i} \qty( \rho_i, \, T) = e.
\end{equation}
где $\rho_i = \beta_i \rho / \alpha_i$. Это система на $\alpha_i$ и $T$, давления здесь вообще нет. Число независимых уравнений равно $m + 1$. Если система уравнений имеет единственное решение, тогда каждое уравнение состояния $P^i(\rho_i, \, T)$ в результате выдаст одинаковое давление. Поскольку $\beta_i$ и $\rho$ известны, можно считать, что заданы функции 
$$P^{i} \qty(\alpha_i, \, T) = P^{i}(\beta_i \rho / \alpha_i, \, T),
\qquad
e^{i} \qty(\alpha_i, \, T) = e^{i}(\beta_i \rho / \alpha_i, \, T).$$

Помимо начального приближения $\alpha^{0}$ добавляется $T^{0}$. Последние два уравнения с суммами распишем по Ньютону:
\begin{equation}
\begin{aligned}
&\sum_i \Delta \alpha_i = 0, \\
&\sum_i \beta_i e^{i}_{\alpha} \Delta \alpha_i = e - e^{\beta} - e^{\beta}_{T} \Delta T,
\end{aligned}
\end{equation}
где $e^{\beta}$ --- внутренняя энергия смеси как функция $(\rho, \, T)$:
\begin{equation}
e^{\beta} = \sum_i \beta_i e^{i} \qty(\rho_i, \, T), 
\qquad
e^{\beta}_T = \sum_i \beta_i e^{i}_T \qty(\rho_i, \, T).
\end{equation}

Пилим Ньютона для уравнений на давление:
\begin{equation}
P^{i} + P^{i}_{\alpha} \Delta \alpha_i + P^{i}_{T} \Delta T = P^{j} + P^{j}_{\alpha} \Delta \alpha_j + P^{j}_{T} \Delta T .
\end{equation}

Выразим отсюда $\Delta \alpha_i$:
\begin{equation}
\Delta \alpha_i = \frac{1}{P^{i}_{\alpha}} \qty[ P^{j} + P^{j}_{\alpha} \Delta \alpha_j + P^{j}_{T} \Delta T - P^{i} - P^{i}_T \Delta T],
\end{equation}
и добавим множитель:
\begin{equation}
\beta_i e^{i}_{\alpha} \Delta \alpha_i = \frac{\beta_i e^{i}_{\alpha}}{P^{i}_{\alpha}} \qty[P^{j} + P^{j}_{\alpha} \Delta \alpha_j + P^{j}_{T} \Delta T - P^{i} - P^{i}_T \Delta T ].
\end{equation}
Коэффициент перед скобкой обозначим как $\eta_i$:
\begin{equation}
\beta_i e^{i}_{\alpha} \Delta \alpha_i = \eta_i \qty[P^{j} + P^{j}_{\alpha} \Delta \alpha_j + P^{j}_{T} \Delta T - P^{i} - P^{i}_T \Delta T ].
\end{equation}

Просуммируем по $i$, слева получится выражение с энергиями:
\begin{equation}
e - e^{\beta} - e^{\beta}_{T} \Delta T = \qty( \Sigma_i \eta_i ) \qty[P^{j} + P^{j}_{\alpha} \Delta \alpha_j + P^{j}_{T} \Delta T] - \qty( \Sigma_i \eta_i P^{i} ) - \qty( \Sigma_i \eta_i P^{i}_T ) \Delta T.
\end{equation}
Введем обозначения для сумм в скобках:
\begin{equation}
\eta_i = \frac{\beta_i e^{i}_{\rho}}{P^{i}_{\rho}}, \qquad
A_{\eta} = \sum_i \eta_i , \qquad
B_{\eta} = \sum_i \eta_i P^{i}, \qquad
C_{\eta} = \sum_i \eta_i P^{i}_T.
\end{equation}

Кажется, всё не так плохо.
\begin{equation}
e - e^{\beta} - e^{\beta}_{T} \Delta T = A_{\eta} \qty( P^{i} + P^{i}_{\alpha} \Delta \alpha_i + P^{i}_{T} \Delta T) - B_{\eta} - C_{\eta} \Delta T.
\end{equation}

Введем величину $\xi_i = - 1 / P^{i}_{\alpha}$, как и ранее. Умножим всё на $\xi_i$:
\begin{equation}
\xi_i \qty[ e - e^{\beta} - e^{\beta}_{T} \Delta T ] = A_{\eta} \qty[ \xi_i P^{i} - \Delta \alpha_i + \xi_i P^{i}_{T} \Delta T ] - \xi_i B_{\eta} - \xi_i C_{\eta} \Delta T.
\end{equation}

Теперь это всё нужно просуммировать по $i$ и учесть, что $\sum_i \Delta \alpha_i = 0$. Получается
\begin{equation}
\qty(\Sigma_i \xi_i) \qty[ e - e^{\beta} - e^{\beta}_{T} \Delta T ] = A_{\eta} \qty[ \qty( \Sigma_i \xi_i P^{i} ) + \qty( \Sigma_i \xi_i P^{i}_{T} ) \Delta T ] - \qty(\Sigma_i \xi_i) B_{\eta} - \qty(\Sigma_i \xi_i) C_{\eta} \Delta T.
\end{equation}
Вводим аналогичные обозначения для сумм в скобках:
\begin{equation}
\xi_i = \frac{\alpha_i}{\rho_i P^{i}_{\rho}}, \qquad
A_{\xi} = \sum_i \xi_i , \qquad
B_{\xi} = \sum_i \xi_i P^{i}, \qquad
C_{\xi} = \sum_i \xi_i P^{i}_T.
\end{equation}

Ну ничего, уже почти всё:
\begin{equation}
A_{\xi} \qty[ e - e^{\beta} - e^{\beta}_{T} \Delta T ] = A_{\eta} \qty[ B_{\xi} + C_{\xi} \Delta T ] - A_{\xi} B_{\eta} - A_{\xi} C_{\eta} \Delta T.
\end{equation}
Очевидно, отсюда надо выразить $\Delta T$:
\begin{equation}
\Delta T = \frac{A_{\eta} B_{\xi} - A_{\xi} B_{\eta} - A_{\xi} \qty(e - e^{\beta} )}{A_{\xi} C_{\eta} - A_{\eta} C_{\xi} - A_{\xi} e^{\beta}_{T}}.
\end{equation}


Ололо, если немного откатиться
\begin{equation}
A_{\xi} \qty[ P^{i} + P^{i}_{\alpha} \Delta \alpha_i + P^{i}_{T} \Delta T ] = B_{\xi} + C_{\xi} \Delta T.
\end{equation}

\begin{equation}
\Delta \alpha_i = \xi_i \cdot \qty( P^{i} - \frac{B_{\xi}}{A_{\xi}} + \qty(P^{i}_T - \frac{C_{\xi}}{A_{\xi}}) \Delta T ).
\end{equation}


\section{Выкладки}

\subsection{Двучленное уравнение состояния и Stiffened Gas}
\label{app:stiffened}

Стартуем с формулы
\begin{equation}
P(\rho, \, e) = \qty(\gamma - 1) \rho e + c_0^2 \qty(\rho - \rho_0)
\end{equation}
Хотим найти термическое уравнение состояния $P(\rho, \, T)$. Просто так его из $P(\rho, \, e)$ не вывести, требуется хотя бы некоторое допущение. Используем следующее, предполагаем, что $\qty( \dfrac{\partial e}{\partial T} )_{\rho} = c_v = \const$, тогда внутренняя энергия представляется в форме
\begin{equation}
e(\rho, \, T) = c_v T + K( \rho ),
\end{equation}

Попробуем найти каноническое уравнение состояния $e(\rho, \, s)$.
\begin{equation}
\qty( \pdv{e}{s} )_{\rho} = T = \frac{e (\rho, \, s) - K(\rho)}{c_v},
\end{equation}
это что-то вроде ОДУ на $e(\rho, \, s)$, общее решение имеет вид
\begin{equation}
e(\rho, \, s) = A(\rho) \exp \qty( \tfrac{s}{c_v} ) + K(\rho).
\end{equation}

Также известно
\begin{equation}
\qty(\pdv{e}{\rho})_{s} = \frac{P}{\rho^2}.
\end{equation}
Тут уже ОДУ которое должно выполняться тождественно $\forall s$:
\begin{equation}
\rho^2 \qty[ A'(\rho) \exp \qty( \tfrac{s}{c_v} ) + K'(\rho)] = P = \qty(\gamma - 1) \rho \qty[ A(\rho) \exp \qty( \tfrac{s}{c_v} ) + K(\rho)] + c_0^2 \qty(\rho - \rho_0)
\end{equation}
Отсюда можно вычислить $A(\rho)$ и $K(\rho)$, а значит и $e(\rho, \, s)$. Решение содержит только две константы $e_1$ и $e_2$. 
\begin{equation}
A(\rho) = e_1 \exp \qty(\tfrac{s}{c_v}) \qty(\frac{\rho}{\rho_0})^{\gamma - 1},
\qquad
K(\rho) = e_2 \qty(\frac{\rho}{\rho_0})^{\gamma - 1} + \frac{\rho_0 c_0^2}{\gamma \rho} - \frac{c_0^2}{\gamma - 1},
\end{equation}
\begin{equation}
e\qty(\rho, \, s) = \qty[e_1 \exp \qty(\tfrac{s}{c_v}) + e_2] \qty(\frac{\rho}{\rho_0})^{\gamma - 1} + \frac{\rho_0 c_0^2}{\gamma \rho} - \frac{c_0^2}{\gamma - 1}.
\end{equation}

Константа $e_1$ перед экспонентой никак не влияет на формулу для давления. Теперь есть два варианта. 1 это волевым решением положить $e_2 = 0$. Пурурум, получится Stiffened Gas.  
\begin{equation}
e\qty(\rho, \, s) = A \exp \qty(\tfrac{s}{c_v}) \rho^{\gamma - 1} + \frac{P_0}{\rho} + e_0.
\end{equation}

Второй вариант. Выпишем давление:
\begin{multline*}
P = (\gamma - 1) \rho (c_v T + K(\rho)) + c_0^2 (\rho - \rho_0) =
(\gamma - 1) \rho \qty[c_v T + e_2 \qty(\frac{\rho}{\rho_0})^{\gamma - 1} + \frac{\rho_0 c_0^2}{\gamma \rho} - \frac{c_0^2}{\gamma - 1}] + \\ + c_0^2 (\rho - \rho_0) = \qty(\gamma - 1) c_v \rho T + \qty(\gamma - 1) \rho_0 e_2 \qty( \frac{\rho}{\rho_0} )^{\gamma} - \frac{\rho_0 c_0^2}{\gamma} = \\ = \qty(\gamma - 1) c_v \rho T + \frac{\rho_0 c_0^2}{\gamma} \qty[ \frac{\gamma \qty(\gamma - 1) e_2}{c_0^2} \qty( \frac{\rho}{\rho_0} )^{\gamma} - 1].
\end{multline*}
Вторую скобку будем называть <<упругой частью>> давления. Будем выбирать параметр $e_0$ таким образом, чтобы упругая часть давления обращалась в ноль при $\rho = \rho_0$, то есть
\begin{equation}
e_2 = \frac{c_0^2}{\gamma \qty(\gamma - 1)} = - \frac{e_0}{\gamma}.
\end{equation}

Второй вариант канонического уравнения состояния:
\begin{equation}
e\qty(\rho, \, s) = \qty[A \exp \qty(\tfrac{s}{c_v}) - \frac{e_0}{\gamma}] \qty(\frac{\rho}{\rho_0})^{\gamma - 1} + \frac{P_0}{\rho} + e_0.
\end{equation}

Параметр $A$ ни на что не влияет. Пофиг.

\subsection{Ми -- Грюнайзена и Мурнагана}

Будем следовать статье \cite{Heuze12} при выводе общей формы уравнения Ми -- Грюнайзена. В начале 20-го века Ми и Грюнайзен разрабатывали теорию твердых материалов, в которой давление линейная функция энергии. Общее наименование Ми -- Грюнайзена относится к моделям, которые следуют этому допущению. Коэффициент пропорциональности $\Gamma(\rho)$, который связывает энергию и давление называется параметром Грюнайзена.

При исследовании кристаллов они получили, что внутренняя энергия кристаллов складывается из их потенциальной энергии при нулевой температуре (потенциал взаимодействия атомов при нулевой температуре), а также из температурной энергии вибраций, которая растет с температурой
\begin{equation}
e(\rho, \, T) = e_{ref} (\rho) + e_{T} (\rho, \, T),
\end{equation}
Аналогичное разложение используется для давления:
\begin{equation}
P(\rho, \, T) = P_{ref} (\rho) + P_T (\rho, \, T),
\end{equation}
Потенциальная энергия $e_{ref}(\rho)$ также называется холодной энергией или референсным потенциалом, $P_{ref}(\rho)$ --- холодное или референсное давление. Они связаны соотношением
\begin{equation}
P_{ref}(\rho) = \rho^2 e'_{ref} (\rho).
\end{equation}

Референсные кривые при нулевой температуре получаются математически при анализе кристаллических решеток, к примеру, простые формулы приводятся здесь \cite{Lemons99}. Правда математика и теоретическая физика дают очень грубые оценки для констант в уравнениях, так что их потом подбирают в экспериментах. Ну хотя бы вид функций известен.


Температурное давление $P_T(\rho, \, T)$ пропорционально энергии $e_{T} (\rho, \, T)$. Их соотношение не зависит от температуры, а только от вибрационной частоты $\nu$ и плотности $\rho$. Параметр Грюнайзена $\Gamma$ можно ввести как
\begin{equation*}
\Gamma (\rho) = \frac{d \ln \nu}{d \ln \rho}.
\end{equation*}
Исходя из \textit{virial theorem} Грюнайзен получил
\begin{equation}
P(\rho, \, e) - P_{ref} \qty(\rho) = \Gamma (\rho) \rho \, \qty[ e - e_{ref}(\rho) ].
\end{equation}

Что ещё интересного вводится в статье. Ну во первых предположим, что $c_v = \qty(de / dT)_{\rho} = \const$, как часто делаем.
Вот есть формула температуры Дебая
\begin{equation}
T_ref(\rho) = T_0 \cdot \exp \qty[ \int_{\rho_0}^{\rho} \frac{\Gamma(\rho)}{\rho} d \rho ],
\end{equation}
которую будем использовать в качестве референсной кривой.
И известно, свойство, указано в статье \cite{Heuze12}.
\begin{equation}
e - e_{ref}(\rho) = T_{ref}(\rho) \qty[ \sigma (s) - \sigma (s_0)].
\end{equation}
где $\sigma (s)$ -- энтропийная функция.

Откуда берется уравнение состояния из основного раздела? Для референсной кривой выбирается закон Мурнагана.
\begin{equation}
P_{ref} (\rho) = \frac{B}{n} \qty[ \qty( \frac{\rho}{\rho_0} )^{n} - 1].
\end{equation}
Для референсной температуры есть две опции, если предположить $\Gamma = \const$, тогда интегрирование даст
\begin{equation}
T_{ref}(\rho) = T_0 \qty( \frac{\rho}{\rho_0} )^{\Gamma},
\end{equation}
ну то есть как для двучленного уравнения состояния ($\Gamma = \gamma - 1$). Можем также предположить, в \cite{Heuze12} написано, что это стандартное предположение для металлов, что $\Gamma (\rho) \rho = \Gamma \rho_0 = \const$. Тогда после интегрирования получим
\begin{equation}
T_{ref}(\rho) = T_0 \exp \qty[ \Gamma \qty(1 - \frac{\rho_0}{\rho} ) ],
\end{equation}
действительно, такая версия встречается, к примеру, у Уолша и Кристиана \cite{Walsh55}. Во видимому, если выбирать эту версию, то и в основном уравнении следует заменить $\rho$ на $\rho_0$. Последний произвол остался в выборе энтропийной функции. Ну тут выбираем как обычно, откуда это следует, я пока не разобрался. Кажется из того что $c_v = \const$
\begin{equation}
\sigma(s) = \exp \qty( \tfrac{s - s_0}{c_v} ).
\end{equation}
Тогда каноническое уравнение состояния:
\begin{equation}
e = e_{ref}(\rho) + c_v T_{ref}(\rho) \qty( \exp \qty( \tfrac{s - s_0}{c_v} ) - 1).
\end{equation}





\chapter{Классическая газодинамика}

\section{Математическая модель}

\section{Приближенные римановские решатели}

\section{Интегрирование по времени}

\chapter{Многоматериальная газодинамика. Равновесная PT-модель.}



\chapter{Приложения/утилиты}

\section{Геометрия}

\subsection{Сечения прямоугольников}

Уравнение прямой на плоскости имеет вид: 
\begin{equation*}
\vb*{r}: \;\; \qty( \vb*{r}, \, \vb*{n} ) = p \qquad \text{или} \qquad
\qty(\vb*{r} - \vb*{r}_0, \, \vb*{n} ) = 0.
\end{equation*}
Здесь $\vb*{n}$ --- нормаль к прямой, внешняя по отношению к полуплоскости $\qty(\vb*{r}, \, \vb*{n} ) < p$, которую будем называть областью \textit{под прямой}. Вектор $\vb*{r}_0$ --- произвольная точка на прямой. Параметр $p$ --- расстояние от центра координат до прямой со знаком, $p > 0$, если центр координат расположен под прямой, в обратном случае $p < 0$. Если уравнение задано в первой форме, то <<опорную>> точку на прямой всегда можно получить по формуле $\vb*{r}_0 = p \cdot \vb*{n}$.
\smallskip

Прямоугольник со сторонами $a$ и $b$ расположен в центре координат, то есть $|x| < \frac{a}{2}$, $|y| < \frac{b}{2}$. Требуется найти объемную долю $\alpha$, которую отсекает от него прямая $\qty( \vb*{r}, \, \vb*{n} ) = p$. То есть долю прямоугольника, которая располагается под прямой (Рис.~\ref{fig:sections-1}).
\begin{equation*}
\xi = \min \qty(a \cdot |n_x|, \, b \cdot |n_y| ), \qquad \eta = \max \qty(a \cdot |n_x|, \, b \cdot |n_y| ),
\end{equation*}
\begin{equation}\label{eq:sec_ap}
\alpha(p) = \left\{
\begin{aligned}
&\half + \frac{p}{\eta}, \quad &&|p| \le \frac{\eta - \xi}{2}, \\
&H(p) - \frac{\sgn(p)}{2\xi \eta} \qty( |p| - \frac{\xi + \eta}{2} )^2, \quad
&&\frac{\eta - \xi}{2} < |p| < \frac{\eta + \xi}{2}, \\
&H(p), \quad && \frac{\eta + \xi}{2} \le |p|.
\end{aligned}
\right.
\end{equation}
Здесь $H(p)$ --- функция Хевисайда. Последняя строка описывает случаи, когда прямая лежит вне прямоугольника. Очевидно, $\alpha( p \to - \infty) = 0$ и $\alpha (p \to \infty) = 1$.
\smallskip

Обратная функция $p(\alpha)$ позволяет найти прямую с нормалью $\vb*{n}$, которая проходит через прямоугольник $|x| < \frac{a}{2}$, $|y| < \frac{b}{2}$ и отсекает от него заданную объемную долю $\alpha$.
\begin{equation*}
\xi = \min \qty(a \cdot |n_x|, \, b \cdot |n_y| ), \qquad \eta = \max \qty(a \cdot |n_x|, \, b \cdot |n_y| ),
\end{equation*}
\begin{equation}\label{eq:sec_pa}
p\qty(\alpha) =\left\{
\begin{aligned}
&\qty(\alpha - \half) \cdot \eta, \;\; &&\qty| 2 \alpha - 1| \le 1 - \frac{\xi}{\eta}, \\
& \sgn \qty(\alpha - \half) \cdot \qty( \frac{\xi + \eta}{2} 
- \sqrt{\qty(1 - \qty|2 \alpha - 1|) \, \xi \eta }), \;\; &&\qty| 2 \alpha - 1| > 1 - \frac{\xi}{\eta}.
\end{aligned}
\right.
\end{equation}
Функция $p(\alpha)$ удовлетворяет свойству симметрии:
\begin{equation*}
p(1 - \alpha) = - p(\alpha).
\end{equation*}

\begin{figure}[b]
\begin{minipage}{0.21\linewidth}
\includegraphics[height=4.5cm]{images/sections/cells1.pdf}
\caption{\label{fig:sections-1}}
\end{minipage}
\begin{minipage}{0.49\linewidth}
\includegraphics[height=4.5cm]{images/sections/cells2.pdf}
\caption{\label{fig:sections-2}}
\end{minipage}
\begin{minipage}{0.27\linewidth}
\includegraphics[height=4.5cm]{images/sections/cells3.pdf}
\caption{\label{fig:sections-3}}
\end{minipage}
\end{figure}

Через две квадратных ячейки с объемными долями $0 <\alpha_1, \, \alpha_2 < 1$ можно провести единственную прямую, которая отсекает заданные объемные доли (Рис.~\ref{fig:sections-2}). Доля грани $\alpha_{\sigma}$, которая отсекается этой прямой, равна:
{\small
\begin{equation}\label{eq:alphas_origin}
\alpha_{\sigma} = \left\{
\begin{aligned}
&\frac{\alpha_{\min} + \alpha_{\max}}{2}, \quad 
&&3 \alpha_{\min} \ge \alpha_{\max} \; \land \; \alpha_{\min} \ge 3 \alpha_{\max} - 2, \\
&-2 \alpha_{\min} + 2 \sqrt{ \alpha_{\min} \cdot \qty( \alpha_{\min} + \alpha_{\max} )}, \quad 
&&\alpha_{\max} \le \thalf \; \lor \; \alpha_{\min} \cdot \qty(1 - \alpha_{\max}) \ge \qty(\alpha_{\max} - \thalf)^2, \\
&3 - 2\alpha_{\max} - 2 \sqrt{\qty(1 - \alpha_{\max}) \cdot \qty(2 - \alpha_{\min} - \alpha_{\max} )}, \quad 
&&\alpha_{\min} \ge \thalf \; \lor \; \alpha_{\min} \cdot \qty(1 - \alpha_{\max}) \ge \qty(\alpha_{\min} - \thalf)^2, \\
&\frac{\sqrt{\alpha_{\min}}}{\sqrt{1 - \alpha_{\max}} + 
\sqrt{\alpha_{\min}}}, \quad 
&&\text{иначе.}
\end{aligned}
\right.
\end{equation}
}

\noindent Условия в формуле необходимо проверять последовательно сверху вниз. Формула справедлива также при $\alpha_{\min} = 0$ и $\alpha_{\max} = 1$, получаются значения $\alpha_{\sigma} = 0$ и $\alpha_{\sigma} = 1$ соответственно (последняя строка). Значение при $\alpha_{\min} = 0$, $\alpha_{\max} = 1$ не определено. Можно выбирать произвольным образом. В целом это не важно, поскольку нет необходимости восстанавливать интерфейс в заполненных или пустых ячейках.

Как получена формула? Прямая, проходящая через две ячейки и отсекающая объемные доли $\alpha_1$ и $\alpha_2$, должна удовлетворять уравнению:
\begin{equation*}
p(\alpha_1) = p(\alpha_2) + \cos \varphi,
\end{equation*}
где $\varphi$ --- угол между внешней нормалью к прямой и вектором от центра ячейки с объемной долей $\alpha_1$ до центра ячейки с долей $\alpha_2$. Если рассмотреть все возможные конфигурации расположения прямой, то получим формулу \eqref{eq:alphas_origin}. Функция $\alpha_{\sigma} (\alpha_1, \, \alpha_2)$ имеет сложную структуру и кусочно задается в четырёх областях. Функция приведена на Рис.~\ref{fig:alphas}.

Функцию  $\alpha_{\sigma} (\alpha_1, \, \alpha_2)$ следует применять при вычислении нормали к реконструированному интерфейсу в ячейке. Для этого нужно использовать метод Гаусса вычисления градиента объемных долей, при этом на грани брать $\alpha_{\sigma}$. После этого нормаль считается как нормализованный градиент объемной доли. При таком определении нормалей в квадратных ячейках, если реальная граница линейна, то она восстанавливается точно.
\medskip

\begin{figure}
\centering
\begin{minipage}{0.32\linewidth}
\includegraphics[width=\linewidth]{images/sections/as1.png}
\end{minipage}
\hspace{2cm}
\begin{minipage}{0.32\linewidth}
\includegraphics[width=\linewidth]{images/sections/as2.png}
\end{minipage}
\caption{\centering Функция $\alpha_{\sigma} \qty( \alpha_1, \, \alpha_2 )$. На втором графике обозначены области кусочного задания функции. Каждая область --- строка в формуле \eqref{eq:alphas_origin}.}\label{fig:alphas}
\end{figure}


Однако, не стоит использовать $\alpha_{\sigma}$ напрямую при решении задачи переноса. Лучше взять скорректированное значение $\alpha_{\sigma}^{*}$, которое берется как доля вещества от части ячейки, которая переносится за один временной шаг (Рис.~\ref{fig:sections-3}). Пусть вещество \textbf{вытекает} (это важно) из ячейки через грань с внешней нормалью $\vb*{n}_f$, то есть нормальная скорость $v_n = (\vb*{v}, \, \vb*{n}_f) > 0$. Объемная доля вещества в ячейке равна $\alpha$, в ячейке проведена линейная реконструкция интерфейса и проведена нормаль к интерфейсу $\vb*{n}$. Определим косинус угла между $\vb*{n}$ и $\vb*{n}_f$, синус угла достаточно знать с точностью до модуля:
\begin{equation*}
\cos \varphi = \qty(\vb*{n}, \, \vb*{n}_f), \qquad |\sin \varphi| = \sqrt{1 - |\cos \varphi|^2}.
\end{equation*}

Введем также <<локальное число Куранта>>:
$\quad C = \dfrac{\Delta t v_n S}{V} \le 1.$
\newpage

Тогда значение $\alpha_{\sigma}^{*}$ можно рассматривать как функцию трёх аргументов $\alpha^{*}_{\sigma} \qty( \alpha, \, \cos \varphi, \, C)$. Порядок вычисления $\alpha^{*}_{\sigma}$ следующий. Сначала определим несколько вспомогательных параметров $\xi_1$, $\eta_1$, $\xi_2$, $\eta_2$:
\begin{equation*}
\xi_{1} = \min \qty( | \cos \varphi |, |\sin \varphi| ),
\quad
\eta_{1} = \max \qty( | \cos \varphi |, |\sin \varphi| ),
\end{equation*}
\begin{equation*}
\xi_{2} = \min \qty( C \cdot | \cos \varphi |, |\sin \varphi | ),
\quad
\eta_{2} = \max \qty( C \cdot | \cos \varphi |, |\sin \varphi | ),
\end{equation*}

Затем вычислим $p_1$ --- расстояние от центра ячейки до интерфейса:
\begin{equation*}
p_1 =\left\{
\begin{aligned}
&\qty(\alpha - \tfrac{1}{2}) \cdot \eta_1, \;\; &&\qty| 2 \alpha - 1| \le 1 - \frac{\xi_1}{\eta_1}, \\
& \sgn \qty(\alpha - \tfrac{1}{2}) \cdot \qty( \frac{\xi_1 + \eta_1}{2} 
- \sqrt{\qty(1 - \qty|2 \alpha - 1|) \, \xi_1 \eta_1 }), \;\; &&\qty| 2 \alpha - 1| > 1 - \frac{\xi_1}{\eta_1}.
\end{aligned}
\right.
\end{equation*}
Далее $p_2$ --- расстояние от центра прямоугольника, который перетекает в соседнюю ячейку до интерфейса. Если рассматривается единичный квадрат $[0, \, 1]^2$ и считается поток через правую сторону, то этот прямоугольник $[1 - C, \, 1] \times [0, \, 1]$, обозначен пунктиром на Рис.~\ref{fig:sections-3}.
\begin{equation*}
p_2 = p_1 + \half \cdot (C - 1) \cdot \cos \varphi.
\end{equation*}

В итоге, объемная доля отсечения от этого прямоугольника:
\begin{equation*}
\alpha^{*}_{\sigma} = \left\{
\begin{aligned}
&\half + \frac{p_2}{\eta_2}, \quad && |p_2| \le \frac{\eta_2 - \xi_2}{2}, \\
&H(p_2) - \frac{\sgn(p_2)}{2 \xi_2 \eta_2} \cdot \qty( |p_2| - \frac{\xi_2 + \eta_2}{2} )^2, \quad && \frac{\eta_2 - \xi_2}{2} < |p_2| < \frac{\xi_2 + \eta_2}{2}, \\
&H(p_2), \quad && \frac{\xi_2 + \eta_2}{2} \le | p_2 |.
\end{aligned}
\right.
\end{equation*}
Объемную долю $\alpha^{*}_{\sigma}$ уже следует использовать в расчетах для задачи переноса.




\section{Специальные функции}


Мне надоело каждый раз сочинять одни и те же часто используемые функции. Надо записать. Далее функция Хевисайда $\eta(x)$ доопределена нулем в нуле (!), функция знака $\sgn(x)$. Получается, не связана с $\sgn$.
\medskip

\subsection*{Гладкая функция Хевисайда}

Гладкая функция $f(x)$ определена при значениях $x > 0$. Определим для значений $x \ge 0$ функцию $\hat{f} (x)$ следующим образом: $\hat{f}(0) = 0$ и $\hat{f}(x) \equiv f(x)$ при $x > 0$. Если $\eta(x)$ --- функция Хевисайда, которая в точке $x = 0$ принимает значение 0, тогда функцию $\hat{f}(x)$ можно записать как $\hat{f}(x) = \eta(x) f(x)$\footnote{Фактически, такое соглашение использует Горюнов при работе с обобщенными функциями в области $x \ge 0$. Такое определение $\eta(x)$ позволяет занулять значение функции в нуле, а также брать обобщенную производную и выполнять преобразование Лапласа}. При этом, если $\displaystyle \lim_{x \to 0} f(x) \neq 0$, тогда новая функция $\hat{f}(x)$ имеет скачок в точке $x = 0$. Собственно, если имеется функция $f(x)$, которую необходимо доопределить новым значением в точке $x = 0$, но при этом хочется это сделать <<гладко>>, без скачков, то можно воспользоваться гладкой функцией Хевисайда $\eta_{s} (x)$ (smooth), работаем с ней аналогично $\hat{f}(x) = \eta_s(x) f(x)$.

Что мы будем называть гладкой функцией Хевисайда $\eta_s(x)$? Введем два параметра: $x_0$ --- смещение функции Хевисайда, $w > 0$ --- ширина. Функция $\eta_s(x) = 0$ при $x \le x_0$, функция $\eta(x) = 1$ при $x \ge x_0 + w$. На интервале $x \in [x_0, \, x_0 + w]$ функция изменяется непрерывно. Очевидно, при уменьшении ширины $w \to 0$ гладкая функция Хевисайда стремится к обычной $\eta(x)$. Будем в общем случае писать как $\eta_s(x)$ (smooth), подразумевая, что в функции есть ещё пара параметров $(x_0, \, w)$. Хотя в простейшем случае можно полагать, что $x_0 = 0$ и $w = 1$.
\smallskip

Почему-то первой в голову пришел вариант гладкой функции с синусом. Вероятно, потому что изначально я строил сигмоиды, которые являются нечетными функциями.Обозначим функцию как $\eta_{sin}(x)$:
\begin{equation}
\eta_{sin}(x) = \left\{
\begin{aligned}
&0,                       &&\quad x \le x_0, \\
&\sin \frac{\pi (x - x_0)}{2w},  &&\quad x \in (x_0, \, x_0 + w), \\
&1,                       &&\quad x \ge x_0 + w. 
\end{aligned}
\right.
\end{equation}
\begin{equation*}
\eta_{sin}(x) = \eta(\xi) \cdot \qty(1 + \eta \qty(1 - \xi) \cdot \qty( \sin \frac{\pi \xi}{2} - 1) ), \qquad \xi = \frac{x - x_0}{w}.
\end{equation*}
Равна нулю при $x \le x_0$, выходит на единицу на отрезке $[x_0, \, x_0 + w]$. Функция гладкая в точке $x = x_0 + w$. Второй вариант записи удобен при векторизации в python. Производная в $x_0$:
\begin{equation*}
\lim_{x \to x_0} \eta_{sn}'(x) = \frac{\pi}{2w}.
\end{equation*}
\smallskip

Зачем использовать синус, он ведь долго считается. Можно использовать полином (polinomial). Самое простое --- квадратичный. Обозначим как $\eta_{p2}(x)$:
\begin{equation*}
\eta_{p2}(x) = \left\{
\begin{aligned}
&0,                       &&\quad x \le x_0, \\
&\xi \cdot \qty(2 - \xi), &&\quad x \in (x_0, \, x_0 + w), \\
&1,                       &&\quad x \ge x_0 + w. 
\end{aligned}
\right.
\end{equation*}
Функция $\eta_{p2}(x)$ гладкая в точке $x = x_0 + w$. Потом выяснил, что эту функцию легко обобщить. Можно выбрать не квадратичный полином, а полином степени $n$. Обозначим как функцию $\eta_{pn}(x)$:
\begin{equation}
\eta_{pn}(x) = \left\{
\begin{aligned}
&0,                   &&\quad x \le x_0, \\
&1 - \qty(1 - \xi)^n, &&\quad x \in (x_0, \, x_0 + w), \\
&1,                   &&\quad x \ge x_0 + w. 
\end{aligned}
\right.
\end{equation}
\begin{equation*}
\eta_{pn}(x) = \eta(\xi) \cdot \qty(1 - \eta \qty(1 - \xi) \cdot \qty( 1 - \xi )^{n} ), \qquad \xi = \frac{x - x_0}{w}.
\end{equation*}
Функция $\eta_{pn}(x)$ в точке $x = x_0 + w$ имеет $n - 1$ производную равную нулю, то есть у неё высокая степень гладкости при $x > 0$. Производная в нуле для полиномиальных Хевисайдов зависит от степени $n$:
\begin{equation*}
\lim_{x \to x_0} \eta_{pn}'(x) = \frac{n}{w},
\end{equation*}
это позволяет регулировать наклон в точке $x = x_0$.
\smallskip

Если полиномиальная функция всё равно недостаточно резкая, то можно использовать функцию с бесконечной производной в $x_0$. Простейший способ сконструировать такую функцию это использовать извлечение корня. Обозначим класс функций $\eta_{rn}(x)$ (rational): 
\begin{equation}
\eta_{rn}(x) = \left\{
\begin{aligned}
&0,                       &&\quad x \le x_0, \\
& \frac{1}{n} \sqrt[n]{\xi} \cdot \qty(n + 1 - \xi), &&\quad x \in (x_0, \, x_0 + w), \\
&1,                       &&\quad x \ge x_0 + w. 
\end{aligned}
\right.
\end{equation}
\begin{equation*}
\eta_{rn} (x) = \eta(\xi) \cdot \qty( 1 + \eta(1 - \xi) \cdot \qty( \frac{1}{n} \sqrt[n]{|\xi|} \cdot (n + 1 - \xi) - 1) ), \qquad \xi = \frac{x - x_0}{w}.
\end{equation*} 
Функция $\eta_{rn}(x)$ является гладкой в точке сшивки $x = x_0 + w$. Хотя производная в точке $x = x_0$, показатель корня $n$ влияет на резкость функции. При $n \to \infty$ функция стремится к обычному Хевисайду. На практике, чтобы не тратить много вычислительных ресурсов, следует использовать квадратный или кубический корень.


\subsection*{Гладкая функция знака/сигмоиды}

Меня не устраивают классические функции-сигмоиды, поскольку они выходят на значения -1 и 1 только асимптотически. Мне нужны функции, которые достигают значения 1 по модулю при выходе за установленные рамки.

Что будем называть гладкой функцией знака? Нечетная функция относительно точки $x = x_0$, при значениях $|x - x_0| \ge w$ принимает значение по модулю равное единице. Желательно гладкая. Гладкую функцию знака с таким определением можно ввести через определенные ранее гладкие функции Хевисайда $\eta_s(x)$. Если $x_0 = 0$, тогда определим гладкую функцию знака $\sgn_s(x)$ как $\sgn_s(x) = \sgn(x) \eta_s(|x|)$. Внимание, следует помнить, что между гладкими версиями \textbf{не работает} связь $\sgn(x) + 1 = 2 \eta(x)$.

Сигмоида с синусом:
\begin{equation}
\sgn_{sin}(x) = \left\{
\begin{aligned}
&-1,                       &&\quad x \le x_0 - w, \\
&\sin \frac{\pi (x - x_0)}{2w},  &&\quad | x - x_0 | \le w, \\
&+1,                       &&\quad x \ge x_0 + w. 
\end{aligned}
\right.
\end{equation}
\begin{equation*}
\sgn_{sin}(x) = \eta_{\half} \qty(|\xi| - 1) \cdot \sgn(\xi) + \eta_{\half} \qty(1 - |\xi|) \cdot \sin \frac{\pi \xi}{2}, \qquad \xi = \frac{x - x_0}{w}.
\end{equation*}
\begin{equation*}
\lim_{x \to x_0} \eta_{sn}'(x) = \frac{\pi}{2w}.
\end{equation*}

Полиномиальная сигмоида:
\begin{equation}
\sgn_{pn}(x) = 
\sgn(x - x_0) \cdot
\left\{
\begin{aligned}
&1 - \qty(1 - |\xi|)^n, &&\quad |x - x_0| < w, \\
&1,                     &&\quad |x - x_0| \ge w. 
\end{aligned}
\right.
\end{equation}
\begin{equation*}
\sgn_{pn}(x) = \sgn(\xi) \cdot \qty(1 - \eta \qty(1 - |\xi|) \cdot \qty( 1 - |\xi| )^{n} ), \qquad \xi = \frac{x - x_0}{w}.
\end{equation*}
Функция $\sgn_{pn}(x)$ в точках $x = x_0 \pm w$ имеет $n - 1$ производную равную нулю, то есть у неё высокая степень гладкости в точках $x \neq x_0$. Производная в $x = x_0$ для полиномиальной функции знака зависит от степени $n$, что позволяет регулировать наклон:
\begin{equation*}
\lim_{x \to x_0} \sgn_{pn}'(x) = \frac{n}{w}.
\end{equation*}

Сигмоида с корнем:
\begin{equation}
\sgn_{rn}(x) =
\sgn (x - x_0) \cdot
 \left\{
\begin{aligned}
& \frac{1}{n} \sqrt[n]{|\xi|} \cdot \qty(n + 1 - |\xi|), &&\quad |x - x_0| < w, \\
& 1,                       &&\quad |x - x_0| \ge w. 
\end{aligned}
\right.
\end{equation}
\begin{equation*}
\sgn_{rn} (x) = \sgn(\xi) \cdot \qty( 1 + \eta(1 - |\xi|) \cdot \qty( \frac{1}{n} \sqrt[n]{|\xi|} \cdot (n + 1 - |\xi|) - 1) ), \qquad \xi = \frac{x - x_0}{w}.
\end{equation*} 
Функция $\sgn_{rn}(x)$ является гладкой в точках сшивки $x = x_0 \pm w$. Показатель корня $n$ влияет на резкость функции, при $n \to \infty$ функция стремится к обычной функции знака. На практике, для оптимизации вычислений, следует использовать квадратный или кубический корень.

\bibliography{library.bib}


\end{document} 
